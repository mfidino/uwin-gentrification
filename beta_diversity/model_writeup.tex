% Options for packages loaded elsewhere
\PassOptionsToPackage{unicode}{hyperref}
\PassOptionsToPackage{hyphens}{url}
\PassOptionsToPackage{dvipsnames,svgnames*,x11names*}{xcolor}
%
\documentclass[
]{article}
\usepackage{lmodern}
\usepackage{amssymb,amsmath}
\usepackage{ifxetex,ifluatex}
\ifnum 0\ifxetex 1\fi\ifluatex 1\fi=0 % if pdftex
  \usepackage[T1]{fontenc}
  \usepackage[utf8]{inputenc}
  \usepackage{textcomp} % provide euro and other symbols
\else % if luatex or xetex
  \usepackage{unicode-math}
  \defaultfontfeatures{Scale=MatchLowercase}
  \defaultfontfeatures[\rmfamily]{Ligatures=TeX,Scale=1}
\fi
% Use upquote if available, for straight quotes in verbatim environments
\IfFileExists{upquote.sty}{\usepackage{upquote}}{}
\IfFileExists{microtype.sty}{% use microtype if available
  \usepackage[]{microtype}
  \UseMicrotypeSet[protrusion]{basicmath} % disable protrusion for tt fonts
}{}
\makeatletter
\@ifundefined{KOMAClassName}{% if non-KOMA class
  \IfFileExists{parskip.sty}{%
    \usepackage{parskip}
  }{% else
    \setlength{\parindent}{0pt}
    \setlength{\parskip}{6pt plus 2pt minus 1pt}}
}{% if KOMA class
  \KOMAoptions{parskip=half}}
\makeatother
\usepackage{xcolor}
\IfFileExists{xurl.sty}{\usepackage{xurl}}{} % add URL line breaks if available
\IfFileExists{bookmark.sty}{\usepackage{bookmark}}{\usepackage{hyperref}}
\hypersetup{
  pdftitle={This model is wrong. This model is useful.},
  colorlinks=true,
  linkcolor=Maroon,
  filecolor=Maroon,
  citecolor=Blue,
  urlcolor=blue,
  pdfcreator={LaTeX via pandoc}}
\urlstyle{same} % disable monospaced font for URLs
\usepackage[margin=1in]{geometry}
\usepackage{color}
\usepackage{fancyvrb}
\newcommand{\VerbBar}{|}
\newcommand{\VERB}{\Verb[commandchars=\\\{\}]}
\DefineVerbatimEnvironment{Highlighting}{Verbatim}{commandchars=\\\{\}}
% Add ',fontsize=\small' for more characters per line
\usepackage{framed}
\definecolor{shadecolor}{RGB}{248,248,248}
\newenvironment{Shaded}{\begin{snugshade}}{\end{snugshade}}
\newcommand{\AlertTok}[1]{\textcolor[rgb]{0.94,0.16,0.16}{#1}}
\newcommand{\AnnotationTok}[1]{\textcolor[rgb]{0.56,0.35,0.01}{\textbf{\textit{#1}}}}
\newcommand{\AttributeTok}[1]{\textcolor[rgb]{0.77,0.63,0.00}{#1}}
\newcommand{\BaseNTok}[1]{\textcolor[rgb]{0.00,0.00,0.81}{#1}}
\newcommand{\BuiltInTok}[1]{#1}
\newcommand{\CharTok}[1]{\textcolor[rgb]{0.31,0.60,0.02}{#1}}
\newcommand{\CommentTok}[1]{\textcolor[rgb]{0.56,0.35,0.01}{\textit{#1}}}
\newcommand{\CommentVarTok}[1]{\textcolor[rgb]{0.56,0.35,0.01}{\textbf{\textit{#1}}}}
\newcommand{\ConstantTok}[1]{\textcolor[rgb]{0.00,0.00,0.00}{#1}}
\newcommand{\ControlFlowTok}[1]{\textcolor[rgb]{0.13,0.29,0.53}{\textbf{#1}}}
\newcommand{\DataTypeTok}[1]{\textcolor[rgb]{0.13,0.29,0.53}{#1}}
\newcommand{\DecValTok}[1]{\textcolor[rgb]{0.00,0.00,0.81}{#1}}
\newcommand{\DocumentationTok}[1]{\textcolor[rgb]{0.56,0.35,0.01}{\textbf{\textit{#1}}}}
\newcommand{\ErrorTok}[1]{\textcolor[rgb]{0.64,0.00,0.00}{\textbf{#1}}}
\newcommand{\ExtensionTok}[1]{#1}
\newcommand{\FloatTok}[1]{\textcolor[rgb]{0.00,0.00,0.81}{#1}}
\newcommand{\FunctionTok}[1]{\textcolor[rgb]{0.00,0.00,0.00}{#1}}
\newcommand{\ImportTok}[1]{#1}
\newcommand{\InformationTok}[1]{\textcolor[rgb]{0.56,0.35,0.01}{\textbf{\textit{#1}}}}
\newcommand{\KeywordTok}[1]{\textcolor[rgb]{0.13,0.29,0.53}{\textbf{#1}}}
\newcommand{\NormalTok}[1]{#1}
\newcommand{\OperatorTok}[1]{\textcolor[rgb]{0.81,0.36,0.00}{\textbf{#1}}}
\newcommand{\OtherTok}[1]{\textcolor[rgb]{0.56,0.35,0.01}{#1}}
\newcommand{\PreprocessorTok}[1]{\textcolor[rgb]{0.56,0.35,0.01}{\textit{#1}}}
\newcommand{\RegionMarkerTok}[1]{#1}
\newcommand{\SpecialCharTok}[1]{\textcolor[rgb]{0.00,0.00,0.00}{#1}}
\newcommand{\SpecialStringTok}[1]{\textcolor[rgb]{0.31,0.60,0.02}{#1}}
\newcommand{\StringTok}[1]{\textcolor[rgb]{0.31,0.60,0.02}{#1}}
\newcommand{\VariableTok}[1]{\textcolor[rgb]{0.00,0.00,0.00}{#1}}
\newcommand{\VerbatimStringTok}[1]{\textcolor[rgb]{0.31,0.60,0.02}{#1}}
\newcommand{\WarningTok}[1]{\textcolor[rgb]{0.56,0.35,0.01}{\textbf{\textit{#1}}}}
\usepackage{graphicx,grffile}
\makeatletter
\def\maxwidth{\ifdim\Gin@nat@width>\linewidth\linewidth\else\Gin@nat@width\fi}
\def\maxheight{\ifdim\Gin@nat@height>\textheight\textheight\else\Gin@nat@height\fi}
\makeatother
% Scale images if necessary, so that they will not overflow the page
% margins by default, and it is still possible to overwrite the defaults
% using explicit options in \includegraphics[width, height, ...]{}
\setkeys{Gin}{width=\maxwidth,height=\maxheight,keepaspectratio}
% Set default figure placement to htbp
\makeatletter
\def\fps@figure{htbp}
\makeatother
\setlength{\emergencystretch}{3em} % prevent overfull lines
\providecommand{\tightlist}{%
  \setlength{\itemsep}{0pt}\setlength{\parskip}{0pt}}
\setcounter{secnumdepth}{-\maxdimen} % remove section numbering
\usepackage{float}

\title{This model is wrong. This model is useful.}
\author{}
\date{\vspace{-2.5em}}

\begin{document}
\maketitle

The goal of this multi-species occupancy model is to simultaneously
estimate:

\begin{enumerate}
\def\labelenumi{\arabic{enumi}.}
\tightlist
\item
  Spatial variation in species occupancy along environmental gradients.
\item
  Spatial variation in site-level species richness along environmental
  gradients.
\item
  Spatial variation in pairwise dissimilarity in community composition
  along environmental gradients.
\end{enumerate}

\hypertarget{spatial-variation-in-species-occupancy-along-environmental-gradients}{%
\subsection{Spatial variation in species occupancy along environmental
gradients}\label{spatial-variation-in-species-occupancy-along-environmental-gradients}}

We could just use a multi-species occupancy model for this! To be brief,
we go out to \(i\) in \(1, \cdots, I\) sites to detect \(s\) in
\(1, \cdots, S\) species over \(j\) in \(1, \cdots, J\) repeat surveys.
Let \(w_{i,s}\) be the number of surveys we detected each species and
\(z_{i,s}\) be the latent species presence of species \(s\) at site
\(i\). We model species presence as a Bernoulli rendom variable such
that:

\[z_{i,s} \sim \text{Bernoulli}(\psi_{i,s})\] where

\[\text{logit}(\psi_{i,s}) = \beta_s^{\psi} x_i\] Here, we assume there
are species level random effects for all intercept and slope terms
(e.g., the intercept would look something like
\(\beta_{s,1}^\psi \sim \text{N}(\mu_1^\psi, \sigma_1^\psi)\) where the
1 indicates that this is the intercept,
\(\mu_1^\psi \sim \text{Normal}(0, 100)\) and
\(\sigma_1^\psi \sim \text{Inv-Gamma}(1,1)\)).

The detection model, assuming equal sampling among sites, is:

\[w_{i,s}|z_{i,s} \sim \text{Binomial}(J, \rho_{i,s} \times z_{i,s} )\]
Where in this case we can use site-specific covariates on \(\rho_{i,s}\)
such that:

\[\text{logit}(\rho_{i,s}) = \beta_s^\rho x_i\] And the coefficients can
have the same random effect structure. Note that while I'm using \(x_i\)
in the latent state and detection models, it's only for convenience.
They can have different covariates.

\hypertarget{estimating-spatial-variation-in-alpha-diversity}{%
\subsubsection{Estimating spatial variation in alpha
diversity}\label{estimating-spatial-variation-in-alpha-diversity}}

Often times, alpha and beta diversity can be of more interest, yet there
occupancy models do not especially provide a way to evaluate these
patterns. There are, of course, standard approaches that ecologists
take. To quantify spatial variation in alpha diversity, for example,
using some form of regression (e.g., Poisson, Negative Binomial,etc.)
could be adequate. Let \(r_i\) be the richness of species at site \(i\),
which we derive as the row sum of \(z_{i,s}\).

We could model detection corrected alpha diversity as:

\[  \sum_{s = 1}^S\boldsymbol{z_i} =  r_i \sim \text{Poisson}(\lambda_i)\]
where \[\text{log}(\lambda_i) = \beta^\lambda x_i\]

Giving the coefficients vague priors (e.g.,
\(\beta^\lambda \sim \text{Normal}(0, 100)\)).

\hypertarget{estimating-beta-diversity}{%
\subsubsection{Estimating Beta
diversity}\label{estimating-beta-diversity}}

Quantifying spatial varition in beta diversity is more complicated, but
in general there are two basic approaches. The first is the `raw-data
approach', wherein the enivonmental / geographical variation in beta
diversity is partitioned through some form of canonical analysis (e.g.,
redundancy analysis). The second is a ``distance based'' approach, where
pairwise dissimilarity among sampled locations is measured and then some
form of matrix regression is used to correlate these distances to
environmental / geographical variation.

As a fan of regression based approaches, I was drawn a towards the
distance based approach, as it could theoretically be possible to fold
in such a technique into \texttt{JAGS}. Of those distance based
approaches,
\href{https://onlinelibrary.wiley.com/doi/10.1111/j.1472-4642.2007.00341.x}{generalized
dissimilarity modeling (gdm)} looked promising. Briefly, after
calculating some type of pairwise dissimilarity metric (\(d_{i,j}\)) for
site pair \(i\) and \(j\), the regression (using non-negative least
squares) looks something like:

\[d_{i,j} = a_0 + \sum_{p=1}^n a_p |x_{p,i} - x_{p,j}|\]

where \(a_0\) is an intercept, \(a_p\) are slope terms, and
\(|x_{p,i} - x_{p,j}|\) is the abosolute value of some form of spatial
or environmental distance between sites. All coefficients in this model
are strictly non-negative. In practice, this model actually uses
i-spline basis functions instead of raw covariate values. I-splines
range from 0 to 1 and create multiple ``new'' covariates for each
``raw'' covariate. These transformed covariates always range between 0
and 1 and steadily creep upwards from 0 to 1 in different, monotonically
increasing patterns. Because of this, i-splines ensures that beta
diversity increases with environmental distance (i.e., monotonicity),
but is still very flexible. In other words, generalized dissimilarity
models assume that compositional similarity increases with environmental
distance. The \texttt{gdm} package in \texttt{R} has a lot of the
functions to make i-splines and the like (written in C++). I took the
time to rewrite them in R while I was working through all of this.

A
\href{https://besjournals.onlinelibrary.wiley.com/doi/10.1111/2041-210X.12710}{similar
modeling approach} to standard gdm, which I thought would be easier to
fold into an occupancy model, is to use binomial regression instead. Let
\(y_{i,g}\) be the number of dissimilar species among a pair of sites
and \(n_{i,j}\) be the total number of unique species among a pair of
sites. Therefore \(y_{i,j}/n_{i,j}\) is the proportion of dissimilar
species among site pairs (which is essentially Jaccard's dissimilarity).
We could then model dissimilarity as:

\[y_{i,g} \sim \text{Binomial}(n_{i,g}, \pi_{i,g})\] where \(\pi_{}i,g\)
is the expected dissimilarity.

Then let:

\[\text{logit}(\pi_{i,g}) = a_0 + \sum_{p=1}^n a_p (x_{p,g} - x_{p,g})\]

where \(a_p\) are given priors on the log scale to ensure they are
non-negative (and again we are using i-splines here instead of raw
covariates). Conversely, \(a_0\) can be any value on the logit-scale.

\hypertarget{fitting-the-model-in-jags}{%
\subsection{\texorpdfstring{Fitting the model in
\texttt{JAGS}}{Fitting the model in JAGS}}\label{fitting-the-model-in-jags}}

We could, if we wanted to, fit only the multi-species model and generate
some psuedo-posteriors by fitting a frequentist alpha diversity model
and beta diversity model to each mcmc step of the \(\boldsymbol{Z}\)
matrix. There are other papers we could cite as well that have done
similar things (though that does not mean it is the correct thing to
do). Instead, I wanted to fit all of these at the same time within
\texttt{JAGS}, and condition \(r_i\) on \(\boldsymbol{z_i}\) and
\(d_{i,g}\) on \(\boldsymbol{z_i}\) and \(\boldsymbol{z_g}\). However,
to derive \(r_i\) and \(d_{i,g}\) in \texttt{JAGS} means we need to
specify them as deterministic nodes (i.e., \texttt{\textless{}-}), but
they must also be stochastic nodes (i.e., \texttt{\textasciitilde{}}) in
\texttt{JAGS} if we want them to estimate spatial variation (i.e., have
them be a function of covariates). To address this, I used the Bernoulli
one's trick so that the data is input into the hand-coded likelihood
(which avoids therefore avoids the need to specify \(r_i\) and
\(d_{i,g}\) as stochastic nodes). I coded up the model in \texttt{JAGS}
like so, using nested indexing for \(d_{i,g}\) to make sure we evalulate
every site pair combination once.

\begin{Shaded}
\begin{Highlighting}[]
\NormalTok{model\{}
  \CommentTok{#------------------------------}
  \CommentTok{# multi-species occupancy model}
  \CommentTok{#------------------------------}
  \ControlFlowTok{for}\NormalTok{(site }\ControlFlowTok{in} \DecValTok{1}\OperatorTok{:}\NormalTok{nsite)\{}
    \ControlFlowTok{for}\NormalTok{(species }\ControlFlowTok{in} \DecValTok{1}\OperatorTok{:}\NormalTok{nspecies)\{}
      \CommentTok{# Linear predictor latent state model}
      \KeywordTok{logit}\NormalTok{(psi[site,species]) <-}\StringTok{ }\KeywordTok{inprod}\NormalTok{(}
\NormalTok{        beta_psi[species,],}
\NormalTok{        design_matrix_psi[site,]}
\NormalTok{      )}
      \CommentTok{# latent state z is a Bernoulli random variable}
\NormalTok{      z[site,species] }\OperatorTok{~}\StringTok{ }\KeywordTok{dbern}\NormalTok{(}
\NormalTok{        psi[site,species]}
\NormalTok{      )}
      \CommentTok{# Linear predictor data model}
      \KeywordTok{logit}\NormalTok{(rho[site,species]) <-}\StringTok{ }\KeywordTok{inprod}\NormalTok{(}
\NormalTok{        beta_rho[species,],}
\NormalTok{        design_matrix_rho[site,]}
\NormalTok{      )}
      \CommentTok{# w (observed data) is a binomial process}
\NormalTok{      w[site,species] }\OperatorTok{~}\StringTok{ }\KeywordTok{dbin}\NormalTok{(}
\NormalTok{        rho[site,species] }\OperatorTok{*}\StringTok{ }\NormalTok{z[site,species],}
\NormalTok{        nsurvey}
\NormalTok{      )}
\NormalTok{    \}}
\NormalTok{  \}}
  \CommentTok{#----------------------}
  \CommentTok{# Alpha diversity model}
  \CommentTok{#----------------------}
  \ControlFlowTok{for}\NormalTok{(site }\ControlFlowTok{in} \DecValTok{1}\OperatorTok{:}\NormalTok{nsite)\{}
    \CommentTok{# Derive species richness}
\NormalTok{    r[site] <-}\StringTok{ }\KeywordTok{sum}\NormalTok{(z[site,])}
    \CommentTok{# linear predictor}
    \KeywordTok{log}\NormalTok{(mu_alpha[site]) <-}\StringTok{ }\KeywordTok{inprod}\NormalTok{(}
\NormalTok{      beta_alpha,}
\NormalTok{      design_matrix_alpha[site,]}
\NormalTok{      )}
    \CommentTok{# Poisson likelihood}
\NormalTok{    alpha_lik[site] <-}\StringTok{ }\OperatorTok{-}\NormalTok{mu_alpha[site] }\OperatorTok{+}\StringTok{ }
\StringTok{      }\NormalTok{r[site]}\OperatorTok{*}\KeywordTok{log}\NormalTok{(mu_alpha[site]) }\OperatorTok{-}\StringTok{ }\KeywordTok{logfact}\NormalTok{(r[site])}
    \CommentTok{# rich is a Poisson random variable via ones trick}
\NormalTok{    alpha_ones[site] }\OperatorTok{~}\StringTok{ }\KeywordTok{dpois}\NormalTok{(}\KeywordTok{exp}\NormalTok{(alpha_lik[site])}\OperatorTok{/}\NormalTok{CONSTANT)}
\NormalTok{  \}}
  \CommentTok{#---------------------}
  \CommentTok{# Beta diversity model}
  \CommentTok{#---------------------}
  \ControlFlowTok{for}\NormalTok{(i }\ControlFlowTok{in} \DecValTok{1}\OperatorTok{:}\NormalTok{n)\{ }\CommentTok{# n = number of unique site pair combos}
    \CommentTok{# Get number of dissimilar species between site pairs (y)}
\NormalTok{    y1[i] <-}\StringTok{ }\KeywordTok{sum}\NormalTok{(}
\NormalTok{      (}\DecValTok{1} \OperatorTok{-}\StringTok{ }\NormalTok{z[siteA_id[i],]) }\OperatorTok{*}
\StringTok{        }\NormalTok{z[siteB_id[i],]}
\NormalTok{    )}
    \CommentTok{# Get total richness between site pairs (n)}
\NormalTok{    y2[i] <-}\StringTok{ }\NormalTok{nspecies }\OperatorTok{-}\StringTok{ }\KeywordTok{sum}\NormalTok{(}
\NormalTok{      (}\DecValTok{1} \OperatorTok{-}\StringTok{ }\NormalTok{z[siteA_id[i],]) }\OperatorTok{*}
\StringTok{        }\NormalTok{(}\DecValTok{1} \OperatorTok{-}\StringTok{ }\NormalTok{z[siteB_id[i],])}
\NormalTok{    )}
    \CommentTok{# Linear predictor}
    \KeywordTok{logit}\NormalTok{(pi[i]) <-}\StringTok{ }\NormalTok{b0 }\OperatorTok{+}\StringTok{ }\KeywordTok{inprod}\NormalTok{(}
\NormalTok{      beta_beta,}
\NormalTok{      design_matrix_beta[i,]}
\NormalTok{    )}
    \CommentTok{# code up binomial log likelihood. JAGS does not have a binomial}
    \CommentTok{# coefficient so it's a bit of a beast to code up. First line}
    \CommentTok{#  is literally just the binomial coefficient.}
\NormalTok{    beta_lik[i] <-}\StringTok{ }\NormalTok{(}\KeywordTok{logfact}\NormalTok{(y2[i]) }\OperatorTok{-}\StringTok{ }\NormalTok{(}\KeywordTok{logfact}\NormalTok{(y1[i]) }\OperatorTok{+}\StringTok{ }\KeywordTok{logfact}\NormalTok{(y2[i] }\OperatorTok{-}\StringTok{ }\NormalTok{y1[i]))) }\OperatorTok{+}\StringTok{ }
\StringTok{      }\NormalTok{(y1[i] }\OperatorTok{*}\StringTok{ }\KeywordTok{log}\NormalTok{(pi[i])) }\OperatorTok{+}\StringTok{ }\NormalTok{((y2[i] }\OperatorTok{-}\StringTok{ }\NormalTok{y1[i]) }\OperatorTok{*}\StringTok{ }\KeywordTok{log}\NormalTok{(}\DecValTok{1} \OperatorTok{-}\StringTok{ }\NormalTok{pi[i]))}
    \CommentTok{# y1 is a binomial process via ones trick}
\NormalTok{    beta_ones[i] }\OperatorTok{~}\StringTok{ }\KeywordTok{dbern}\NormalTok{( }
      \KeywordTok{exp}\NormalTok{(beta_lik[i])}\OperatorTok{/}\NormalTok{CONSTANT}
\NormalTok{    )}
\NormalTok{  \}}
  \CommentTok{#-------}
  \CommentTok{# priors}
  \CommentTok{#-------}
  \CommentTok{# Multispecies occupancy}
  \ControlFlowTok{for}\NormalTok{(psii }\ControlFlowTok{in} \DecValTok{1}\OperatorTok{:}\NormalTok{npar_psi)\{}
    \CommentTok{# community mu & sd occupancy}
\NormalTok{    beta_psi_mu[psii] }\OperatorTok{~}\StringTok{ }\KeywordTok{dlogis}\NormalTok{(}\DecValTok{0}\NormalTok{,}\DecValTok{1}\NormalTok{)}
\NormalTok{    tau_psi[psii] }\OperatorTok{~}\StringTok{ }\KeywordTok{dgamma}\NormalTok{(}\FloatTok{0.001}\NormalTok{,}\FloatTok{0.001}\NormalTok{)}
\NormalTok{    sd_psi[psii] <-}\StringTok{ }\DecValTok{1} \OperatorTok{/}\StringTok{ }\KeywordTok{sqrt}\NormalTok{(tau_psi[psii])}
    \CommentTok{# community my & sd detection}
\NormalTok{    beta_rho_mu[psii] }\OperatorTok{~}\StringTok{ }\KeywordTok{dlogis}\NormalTok{(}\DecValTok{0}\NormalTok{,}\DecValTok{1}\NormalTok{)}
\NormalTok{    tau_rho[psii] }\OperatorTok{~}\StringTok{ }\KeywordTok{dgamma}\NormalTok{(}\FloatTok{0.001}\NormalTok{,}\FloatTok{0.001}\NormalTok{)}
\NormalTok{    sd_rho[psii] <-}\StringTok{ }\DecValTok{1} \OperatorTok{/}\StringTok{ }\KeywordTok{sqrt}\NormalTok{(tau_rho[psii])}
    \CommentTok{# Species specific coefficients}
    \ControlFlowTok{for}\NormalTok{(species }\ControlFlowTok{in} \DecValTok{1}\OperatorTok{:}\NormalTok{nspecies)\{}
\NormalTok{      beta_psi[species,psii] }\OperatorTok{~}\StringTok{ }\KeywordTok{dnorm}\NormalTok{(}
\NormalTok{        beta_psi_mu[psii],}
\NormalTok{        tau_psi[psii]}
\NormalTok{      )}
\NormalTok{      beta_rho[species,psii] }\OperatorTok{~}\StringTok{ }\KeywordTok{dnorm}\NormalTok{(}
\NormalTok{        beta_rho_mu[psii],}
\NormalTok{        tau_rho[psii]}
\NormalTok{      )}
\NormalTok{    \}}
\NormalTok{  \}}
  \CommentTok{# Alpha diversity}
  \ControlFlowTok{for}\NormalTok{(alphai }\ControlFlowTok{in} \DecValTok{1}\OperatorTok{:}\NormalTok{npar_alpha)\{}
\NormalTok{    beta_alpha[alphai] }\OperatorTok{~}\StringTok{ }\KeywordTok{dnorm}\NormalTok{(}\DecValTok{0}\NormalTok{,}\FloatTok{0.01}\NormalTok{)}
\NormalTok{  \}}
  \CommentTok{# beta diversity}
  \ControlFlowTok{for}\NormalTok{(betai }\ControlFlowTok{in} \DecValTok{1}\OperatorTok{:}\NormalTok{npar_beta)\{}
\NormalTok{    beta_log[betai] }\OperatorTok{~}\StringTok{ }\KeywordTok{dnorm}\NormalTok{(}\DecValTok{0}\NormalTok{, }\FloatTok{0.01}\NormalTok{)}
\NormalTok{    beta_beta[betai] <-}\StringTok{  }\KeywordTok{exp}\NormalTok{(beta_log[betai])}
\NormalTok{  \}}
\NormalTok{  b0 }\OperatorTok{~}\StringTok{ }\KeywordTok{dlogis}\NormalTok{(}\DecValTok{0}\NormalTok{,}\DecValTok{1}\NormalTok{) }\CommentTok{# beta diversity intercept}
\NormalTok{\}}
\end{Highlighting}
\end{Shaded}

\hypertarget{comparing-alpha-diversity-outputs-to-other-techniques}{%
\subsection{Comparing alpha diversity outputs to other
techniques}\label{comparing-alpha-diversity-outputs-to-other-techniques}}

To do this, I simualted data for 50 sites and 30 species with the
multi-species model and a single environmental gradient that the
community, on average, responded positively too. As such, we should
expect species richness to increase along this environmental gradient
(and it does). Having 30 species in this case ensured that there was
always at least 1 species present (because dealing with this is a whole
other can of worms).

To compare the alpha the outputs from this Bayesian model I fit two
other models to the raw data (which is what we would have done if we did
not use this model):

1.A ``Truth'' model. A Poisson glm fit to the simulated (but unknown)
true species richness at these sites as a function of the environmental
gradient (i.e.,
\texttt{glm(r\ \textasciitilde{}\ x,\ family\ =\ "poisson"})). This
isn't absolute truth, of course, but it's the best fit line based on the
data that is ``impossible'' for us to observe. 2. An ``Observed'' model.
A Poisson glm fit to the simulated but observed species richness (i.e,.
there was imperfect detection). Note that I specified that community
level detectability was lower with increasing x. In other words, while
species richness went up with x, detectability went down.

And here is what the results look like:

\begin{figure}
\centering
\includegraphics{"alpha_comparison.jpeg"}
\caption{alpha diversity model comparison}
\end{figure}

Based on this single simulation, the detection corrected model slightly
overestimates species richness at the negative end of the environmental
gradient. If x = -2.75, the Bayesian model estimated species richness as
8.3 (95\% CI = 6.5 - 10.5), the truth model estimated species richness
as 6.4 (95\% CI = 5.0 - 8.2), and the observed model estimated species
richness as 7.1 (95\% CI = 5.5, 9.1). Credible interval width was
highest for the detection corrected model (about 4, 3.2, and 3.6
respectively for the detection corrected, truth, and observed models).
At the positive end of this gradient (x = 2.75), the detection corrected
model slightly underestimated species richness by about 1 species from
the true model. Credible interval width was comparable between the
detection corrected model and the true model. The observed model
performed poorly at positive values of the environmental gradient
(probably owing to the fact that detection probability was lower on
average with positive values of the gradient).

\hypertarget{comparing-beta-diversity-outputs-to-other-techniques}{%
\subsection{Comparing beta diversity outputs to other
techniques}\label{comparing-beta-diversity-outputs-to-other-techniques}}

Comparing beta diversity to frequentist models of a gdm is more
difficult because you cannot fit a standard glm (i.e., the likelihood
function is different because of the logged slope terms). Fortunately,
the (now defunctional) \texttt{bbgdm} package has the exact model I
wanted to use, so I cannibalized a bunch of the code there to create the
frequentist gdm model that uses binomial regression. So again, we have
the gdm fit via the bayesian model, a ``truth'' model, and an
``observed'' model for comparison. To get confidence intervals and the
like I applied a bootstrap (swapping pairs of site in the z matrix).
Since there are 50 sites there are only 1225 possible combinations of
pairwise site permutations, thus I just did all of them.

A standard way to demonstrate change with gdms is to make a prediction
on the link scale (i.e., logit) dissimilarity on the y axis and the
environmental gradient on the x axis. This is for one covariate (which
was 3 splines in the model), and excludes the model intercept.

\begin{figure}
\centering
\includegraphics{"beta_comparison.jpeg"}
\caption{beta diversity model comparison}
\end{figure}

So what do we see? That results from detection corrected model is more
similar to the ``true'' model than the ``observed'' model, which is what
we'd expect. It seems as if the detection corrected model (in this
simulation) slightly underpredicted dissimilarity at the positive end of
the gradient, but not by much. 95\% CI width were a comperable among
models, and the ``truth'' (i.e,. the beta dissimilarity estimated from
the model fit to the true data) fell within the 95\% CI of the Bayesian
model but not the observed richness model.

\hypertarget{my-conclusions-on-this-model}{%
\subsection{My conclusions on this
model}\label{my-conclusions-on-this-model}}

Even with the triple dipping we are doing here, this model does better
than what we would normally do: fit the models to the observed data and
hope for the best. So is this model useful? Yes. Is this model also
wrong because it assumes independence among models? Also yes. But, and
this is the big but, is that it gets us closer results to ``truth'' than
the standard models that would be fit.

\hypertarget{moving-forward}{%
\subsection{Moving forward}\label{moving-forward}}

I still wonder if we need to model \(\psi\) and \(\rho\) within the same
linear predictor? This could create the same problem as before, but we
could derive expected occupancy as \(\psi \times (1 - (1-\rho)^J)\)
(i.e,. the probability of occupancy multiplied by the probability you
detect the species at least once in \(J\) surveys)? Anyways, now that I
have this whole idea flushed out on paper I'd love to hear your
thoughts!

\end{document}
